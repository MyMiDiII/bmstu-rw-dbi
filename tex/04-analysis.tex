\chapter{Анализ предметной области}

\section{Основные определения}

\textit{Индекс} --- это некоторая структура, обеспечивающая быстрый поиск
записей в базе данных~\cite{amur}. Индекс определяет соответствие значения
аттрибута или набора аттрибутов конкретной записи с местоположением этой записи.
Аттрибут или набор аттрибутов, по которым осуществляется поиск записей
называется \textit{ключом поиска}.

Каждый индекс связан с определенной таблицей, но не является обязательной ее
составляющей, и поэтому обычно хранится отдельно и не влияет на размещение
данных в табилце.

Если говорить конкретнее, индекс --- это файл с индексными записями. Файл
индекса имеет одну индексную запись для каждой записи в соответствующей таблице.
Каждая индексная запись имеет два поля, которые хранят идентификатор
соответствующей записи в таблице и значение индексированного поля в этой
записи~\cite{syore}.

Обеспечение уменьшения времени доступа к записям по значению ключа достигается
за счет:
\begin{itemize}
    \item упорядочивания значений ключа поиска, что уменьшает количество
        записей, которые необходимо просмотреть;
    \item а также меньшего размера индекса по сравнению с индексируемой
        таблицей, что сокращает время чтения одного элемента.
\end{itemize}

Индекс, хотя и обеспечивает выигрыш в скорости доступа к данным в БД, явлется
структурой, которая строится в дополнение к существующим данным, то есть она
занимает дополнительный объем памяти и должна соответствовать текущим данным,
что требует ее поддержания в актуальном состоянии. Последнее значит, что индекс
необходимо изменять при вставке или удалении элементов, что может замедлить
работу СУБД.

Таким образом, можно выделить следующие характеристики индексов~\cite{ship}:

\begin{itemize}
    \item \textit{тип доступа} --- поиск записей по аттрибуту с конкретным
        значением, или со значением из указанного диапазона;
    \item \textit{время доступа} --- время поиска записи или записей;
    \item \textit{время вставки}, включающее время поиска правильного места вставки, а
        также время для обновления индекса;
    \item \textit{время удаления}, аналогично вставке, включающее время на поиск
        удаляемого элемента и время для обновления индекса;
    \item \textit{дополнительная память}, занимаемая индексной стркутурой.
\end{itemize}

\section{Типы индексов}

Типы индексов выделяют по нескольким признаками. По \textit{типу ключа поиска}
индексы делятся на: 
    \begin{itemize}
        \item первичные --- по первичному ключу ???,
        \item вторичные --- по всем остальным аттрибутам;
    \end{itemize}

По порядку записей в индексируемой таблице индексы делятся на кластеризованные и
некластеризованные.  В \textit{кластеризованных} индексах логический порядок
ключей определяет физическое расположение записей, а так как строки в таблице
могут быть упорядочены только в одном порядке, то кластеризованный индекс может
быть только один на таблицу. Логический порядок \textit{некластеризованных}
индексов не влияет на физический, и индекс содержит указатели на записи
таблицы.

Также индексы делятся на плотные и разрженные. Плотные индексы содержат ключ
поиска и указатель на первую запись с заданным ключом поиска. При этом в
кластеризованных индексах другие записи с заданным ключом будут лежать сразу после
первой записи, так как записи в таких файлах отсортированны по тому же ключу.
Плотные некластеризованные индексы должны содержать список указателей на каждую
запись с заданным ключом поиска.(рисунок~\ref{img:dense}).
 
\imgs{dense}{h!}{0.4}{Плотный индекс}

В разреженных индексах записи содержат только некоторые значения ключа поиска, а
для доступа к элементу отношения ищется запись индекса с наибольшим меньшим или
равным значением ключа поиска, происходит переход по указателю на первую запись
по найденному ключу и далее по указателям в файле происходит поиск заданной
записи. Таким образом, разреженные индексы могут быть построены только на
отсортированных последовательностях записей, иначе хранения только некоторых
ключей поиска будет недостаточно, так как будет неизвестно, после записи, с
каким ключом будет лежать необходимый элемент отношения.
(рисунок~\ref{img:sparse});

\imgs{sparse}{h!}{0.4}{Разреженный индекс}

Поиск с помощью неразреженных индексов быстрее, так как указатель в записи
индекса сразу приводит к необходимым записям. Однако разреженные индексы требуют
меньше дополнительной памяти и сокращают время поддержания структуры индекса в
актуальном состоянии при вставке или удалении.

По количеству уровней:
\begin{itemize}
    \item одноуровневые ... (рисунок~\ref{img:onelevel}),
    \item многоуровневые ... (рисунок~\ref{img:multilevel});
\end{itemize}

По структуре индексы подразделяются на
\begin{itemize}
    \item упорядоченные, на основе деревьев поиска,
    \item хеш-индексы,
    \item индексы, на основе битовых карт.
\end{itemize}

Построение структур каждого из приведенных типов индекса рассматривается в
отдельном разделе, так как именно оно исследуется в данной работе.

