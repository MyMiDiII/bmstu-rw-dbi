\chapter{Анализ предметной области}

\section{Основные определения}

\bfit{Индекс} --- это структура данных, которая определяет соответствие значения
аттрибута или набора аттрибутов конкретной записи с местоположением этой записи.
Аттрибут или набор аттрибутов, по которым осуществляется поиск записей
называется \bfit{ключом поиска}.

Каждый индекс связан с определенной таблицей, но не является обязательной ее
составляющей, и поэтому обычно хранится отдельно и не влияет на размещение
данных в табилце.

Основная цель индекса --- обеспечение уменьшения времени доступа к записям по
значению ключа, которое достигается за счет:

\begin{itemize}
    \item упорядочивания значений ключа поиска, что уменьшает количество
        записей, которые необходимо просмотреть;
    \item а также меньшего размера индекса по сравнению с индексируемой
        таблицей, что сокращает время чтения одного элемента.
\end{itemize}

\section{Проблемы, возникающие при использовании индексов}

Хотя индекс уменьшает время доступа к записям, его использование влечет за собой
проблемы, которые стоит учитывать. Как было сказано выше, индекс представляет
собой структуру, которая строится в дополнение к существующим данным, то есть
она занимает дополнительный объем памяти и должна соответствовать текущим
данным. Таким образом, необходимо изменять данную структуру при вставке или
удалении элементов, что может замедлить работу СУБД.

Таким образом, можно выделить следующие характеристики индексов:

\begin{itemize}
    \item \bfit{тип доступа} --- поиск записей по аттрибуту с конкретным
        значением, или со значением из указанного диапазона;
    \item \bfit{время доступа} --- время поиска записи или записей;
    \item \bfit{время вставки}, включающее время поиска правильного места вставки, а
        также время для обновления индекса;
    \item \bfit{время удаления}, аналогично вставке, включающее время на поиск
        удаляемого элемента и время для обновления индекса;
    \item \bfit{дополнительная память}, занимаемая индексной стркутурой.
\end{itemize}

\section{Типы индексов}

\subsection{По структуре}

\begin{itemize}
    \item упорядоченные, на основе деревьев поиска;
    \item хеш-индексы;
    \item индексы, на основе битовых карт.
\end{itemize}

Построение структур каждого из приведенных типов индекса рассматривается в
отдельном разделе, так как именно оно исследуется в данной работе.

\subsection{По типу ключа поиска}

\begin{itemize}
    \item первичные --- по первичному ключу;
    \item вторичные --- по всем остальным аттрибутам.
\end{itemize}

\subsection{По порядку записей в индексируемой таблице}

\begin{itemize}
    \item кластеризованные ...;
    \item некластеризованные ... .
\end{itemize}

\subsection{По индексируемым значениям}
Плотные и разреженные индексы

\begin{itemize}
    \item плотные ... (рисунок~\ref{img:dense});
    \item разреженные ... (рисунок~\ref{img:sparse}).
\end{itemize}

\subsection{По количеству уровней}

\begin{itemize}
    \item одноуровневые ... (рисунок~\ref{img:onelevel});
    \item многоуровневые ... (рисунок~\ref{img:multilevel}).
\end{itemize}


