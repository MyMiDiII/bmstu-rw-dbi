\chapter{Анализ предметной области}

Данных много => актуально => базы данных.

База данных --- это ...

Основная операция --- поиск => создание методов для ускорения данной операции,
одним из которых является индексы (\bfit{есть ли другие???}).

Индекс -- это ...

Существует два основных вида индексов \bfit{уточнить???}:

\begin{itemize}
    \item упорядоченные, реализующиеся на основе деревьев поиска;
    \item хеш-индексы, в которых поиск значений осуществляется с помощью
        вычисления хеш-функции.
    \item \bfit{bitmap-индексы??? (индексы на основе битовых карт).}
\end{itemize}

Индекс представляет собой структуру, которая строится в дополнение к
существующим данным. Таким образом, она занимает дополнительный объем памяти и
должна соответствовать текущим данным, то есть необходимо изменять данную
структуру при вставке или удалении элементов. Так как индексы создаются для
осуществления поиска, то они также характеризуются типом и временем доступа.

Таким образом, можно выделить следующие характеристики индексов (\bfit{мб по ним
и оценивать, почему нет}):


\begin{itemize}
    \item тип доступа ---;
    \item время доступа ---;
    \item время вставки ---;
    \item время удаления ---;
    \item дополнительная память --- ;
\end{itemize}
