\chapter*{ВВЕДЕНИЕ}
\addcontentsline{toc}{chapter}{ВВЕДЕНИЕ}

На протяжении последнего десятилетия происходит автоматизация все большего числа
сфер человеческой деятельности~\cite{koptenok}. Это приводит к тому, что с
каждым годом производится все больше данных. Так, по исследованию компании
IDC~(International Data Corporation), занимающейся изучением мирового рынка
информационных технологий и тенденций развития технологий, объем данных к
2025~году составит около 175~зеттабайт, в то время как на год исследования их
объем составлял 33~зеттабайта~\cite{idc}. При этом данные необходимо хранить и
обрабатывать.

Хранятся в базе данных. Поступают запросы, которые обрабатывает СУБД. Поиск.
Уменьшение времени обработки запроса -> несколько методов -> один из них
индексы. 1. 

Поэтому проводятся исследования для уменьшения временных и пространственных
сложностей. Так, в 2018~году авторами статьи~\cite{main} было проведено
исследование...

%была опубликована статья <<The Case for Learned
%Index Structures>>, положившая начало исследованиям обученных индексов (learned
%indexes)

%Оптимизация запросов является важной частью работы любого приложения,
%взаимодействующего с базами данных.
%
%Практически во всех современных веб-приложениях эффективные способы доступа к
%обработке данных являются критически важными задачами. Как следствие, в
%проектах, взаимодействующих с базами данных, следует уделять внимение
%оптимизации запросов, иначе время отклика системы на запросы пользователей
%становится неприемлемым.
%
%К самым распространенным из них относятся: изучение плана выполнения запросы,
%индексирование полей реляционных таблиц и анализ степени избирательность
%индексов~\cite{bits}.
%
%1. Индексирование данных заключается в выполнении их предварительной обработки
%с целью более быстрого выполнения в дальнейшем многократных поисковых
%запросов~\cite{newpsindex}.
%
%Без надлежащей индексации СУБД вынуждена будет всякий раз заново сканировать всю
%таблицу в поисках за- прошенных данных.~\cite{dama}.
%
%Основная цель базы данных - управлять данными и сделать
%ускорить поиск данных, а также ускорить вставку, удаление и
%обновление [1]. Правильное индексирование помогает
%в уменьшении временных и пространственных сложностей при хранении и
%поиска~\cite{bigdata}.
%
%Мир данных правит нами с последнего десятилетия. Существуют различные проблемы,
%связанные с эффективным хранением, обработкой и управлением увеличивающимися с
%каждым днем разнородными данными. Отсюда возникает проблема обработки запросов,
%которая помогает эффективно извлекать данные из базы данных с помощью различных
%методов индексирования за меньшее время~\cite{compare2020}.

Почему актуальны индексы?

Большие данные -> много запросов -> необходимость быстрого поиска -> индексы.

Машинное обучение в индексах.

В 2018 году статья Learned Indexes -> показали уменьшение времени поиска ->
заинтерисованность научного сообщества -> \red{много} статей
\cite{1}\cite{2}\cite{3}...

Так как Learned Indexes используют идеи базовых индексных структур: деревья
поиска, хеш-таблицы, битовые карты.

Поэтому в этой работе для каждой из вышеперечисленных структур сначала
описывается построения индекса на основе ее, а далее приводится описание
соответсвтующего обученного индекса.

Целью данной работы является \bfit{классификация методов построения индексов в
базах данных}.

Для достижения поставленной цели требуется решить следующие задачи:
\begin{itemize}
    \item описываются методы построения индексов в базах данных;
    \item предлагаются и обосновываются критерии оценки качества описанных методов;
    \item сравниваются методы по предложенным критериям оценки;
    \item выделяются методы, показывающие лучшие результаты по одному или
        нескольким критериям.
\end{itemize}
