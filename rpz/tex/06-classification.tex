\chapter{Классификация существующих методов\label{classification}}

Индексы в соответствии со структурой лежащей в их основе служат для различных
задач: индексы на основе деревьев поиска предназначены для поиска ключей в
некотором диапазоне, индексы на основе хеш-таблиц --- для поиска единичных
ключей, индексы на основе битовых карт --- для проверки существования ключа.
Поэтому ниже описываются критерии оценки качества методов построения индексов
отдельно для каждой из решаемых задач.

Для оценки качества методов построения индексов для поиска ключей, принадлежащих
некоторому диапазону, выделены следующие критерии:
\begin{itemize}
    \item временная сложность поиска и вставки;
    \item время выполнения поиска и вставки~(в наносекундах);
    \item среднее количество обращений к памяти при поиске и вставке.
\end{itemize}

В таблицах~\ref{tab:01}-\ref{tab:02} приведены результаты классификации по
описанным критериям для поиска и вставки соответственно. Значения в таблицах
представлены на основе исследования~\cite{ulipp}.

{
\captionsetup{format=hang,justification=raggedleft,
              singlelinecheck=off,width=17cm}
\begin{longtable}[Hc]{|p{5cm}|p{2cm}|p{2cm}|p{2cm}|}
\caption{Классификация методов построения индексов для поиска ключей,
принадлежащих некоторому диапазону (поиск)\label{tab:01}}\\
    \hline
    \multicolumn{1}{|c|}{\textbf{Метод}} &
    \multicolumn{1}{c|}{\textbf{Сложность}} &
    \multicolumn{1}{c|}{\textbf{Время, нс}} &
    \multicolumn{1}{c|}{\parbox{3cm}{\vspace{2mm}\centering\textbf{Обращения к
    памяти}}}\\[2.5ex]
    \hline
    B-деревья & $O(\log N)$
    & $237.94$
    & $57.0$\\
    \hline
    Обученные индексы & $O(\log N)$
    & $139.09$
    & $12.6$\\
    \hline
    LIPP & $O(\log N)$
         & \color{white}1\color{black}$24.23$
    & \color{white}1\color{black}$3.1$\\
    \hline
\end{longtable}
}

{
\captionsetup{format=hang,justification=raggedleft,
              singlelinecheck=off,width=17cm}
\begin{longtable}[Hc]{|p{5cm}|p{2cm}|p{2cm}|p{2cm}|}
\caption{Классификация методов построения индексов для поиска ключей,
принадлежащих некоторому диапазону (вставка)\label{tab:02}}\\
    \hline
    \multicolumn{1}{|c|}{\textbf{Метод}} &
    \multicolumn{1}{c|}{\textbf{Сложность}} &
    \multicolumn{1}{c|}{\textbf{Время, нс}} &
    \multicolumn{1}{c|}{\parbox{3cm}{\vspace{2mm}\centering\textbf{Обращения к
    памяти}}}\\[2.5ex]
    \hline
    B-деревья
    & $O(\log N)$
    & $1114.19$
    & $57.8$\\
    \hline
    Обученные индексы
    & ---
    & ---
    & ---\\
    \hline
    LIPP 
    & $O(\log^2 N)$
    & \color{white}11\color{black}$70.93$
    & \color{white}1\color{black}$3.1$\\
    \hline
\end{longtable}
}
~\\

Таким образом, обученные индексы~(с учетом отнесения к ним LIPP) могут
обеспечить уменьшение времени при поиске в~9.8~раза, при вставке ---
в~15.7~раза, и уменьшить число обращений к памяти в~18.6~раза.

Для оценки качества методов построения индексов для поиска единичных выделены
следующие критерии:
\begin{itemize}
    \item временная сложность поиска в худшем и среднем случае~\cite{main,
        squares};
    \item процент коллизий~(значения на основе исследования~\cite{main}),
        представляющий отношения количества коллизий к числу записей.
        
\end{itemize}

В таблице~\ref{tab:03} приведены результаты классификации по описанным
критериям.

{
\captionsetup{format=hang,justification=raggedleft,
              singlelinecheck=off,width=17cm}
\begin{longtable}[Hc]{|p{5.3cm}|p{2cm}|p{2cm}|p{2cm}|}
\caption{Классификация методов построения индексов для поиска единичных
ключей\label{tab:03}}\\
    \hline
    \multicolumn{1}{|c|}{\multirow{2}{*}{\textbf{Метод}}} &
    \multicolumn{2}{c|}{\textbf{Сложность}} &
    \multicolumn{1}{c|}{\multirow{2}{*}{\parbox{2cm}{\textbf{Процент\newlineколлизий}}}}\\
    \cline{2-3}
    & \multicolumn{1}{c|}{\textbf{Худший}}
    & \multicolumn{1}{c|}{\textbf{Средний}}
    &\\
    \hline
    Хеш-индексы
    & $O(N)$
    & $O(1)$
    & $35.3\%$\\
    \hline
    Обученные хеш-индексы
    & $O(N)$
    & $O(1)$
    & $19.5\%$\\
    \hline
\end{longtable}
}

Таким образом, обученные хеш-индексы могут обеспечить уменьшение количества
коллизий на~$44.8\%$.

Для оценки качества методов построения индексов для проверки существования ключа
выделены следующие критерии~(на основе исследования~\cite{main}):
\begin{itemize}
    \item временная сложность поиска;
    \item размер индекса на одном и том же наборе данных~(в мегабайтах).
\end{itemize}

В таблице~\ref{tab:04} приведены результаты классификации по описанным
критериям~($k$ --- количество хеш-функций).

{
\captionsetup{format=hang,justification=raggedleft,
              singlelinecheck=off,width=17cm}
\begin{longtable}[Hc]{|p{5.3cm}|p{2cm}|p{2cm}|}
\caption{Классификация методов построения индексов для проверки существования
ключа в наборе данных\label{tab:04}}\\
    \hline
    \multicolumn{1}{|c|}{\textbf{Метод}} &
    \multicolumn{1}{c|}{\textbf{Сложность}} &
    \multicolumn{1}{c|}{\textbf{Размер, MБ}}\\
    \hline
    Фильтр Блума
    & $O(k)$
    & $2.04$\\
    \hline
    Обученные индексы
    & $O(1)$
    & $1.31$\\
    \hline
\end{longtable}
}

Таким образом, обученные индексы проверки существования ключа в наборе
данных могут обеспечивать уменьшение своего размера на~$36\%$.
