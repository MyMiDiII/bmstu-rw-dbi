\chapter{Классификация существующих методов\label{classification}}

На основе приведенных выше описаний можно сделать вывод, что индексы в
соответствии со структурой лежащей в их основе служат для различных задач:
индексы на основе деревьев поиска предназначены для поиска ключей, принадлежащих
некоторому диапазону, индексы на основе хеш-таблиц --- для поиска единичных
ключей, индексы нас основе битовых карт --- для проверки существования ключа.
Поэтому ниже описываются критерии оценки качества методов построения
индексов отдельно для каждой из задач, которые они решают.

Для оценки качества методов построения индексов на основе хеш-таблиц выделены
следующие критерии:
\begin{itemize}
    \item временная сложность поиска в худшем и среднем случае~\cite{main,
        squares};
    \item процент коллизий~(значения на основе исследования~\cite{main}),
        рассчитывающийся по формуле~\eqref{eq:06}:
        \begin{equation}\label{eq:06}
            C = \frac{k}{n},
        \end{equation}
        
      где $k$ --- число коллизий,
      
      $n$ --- число записей в исходном наборе данных.
\end{itemize}

Результаты классификации по описанным критериям, приведены в таблице~\ref{tab:02}.

{
\captionsetup{format=hang,justification=raggedleft,
              singlelinecheck=off,width=15.3cm}
\begin{longtable}[Hc]{|p{5.3cm}|p{2cm}|p{2cm}|p{2cm}|}
\caption{Классификация методов построения индексов для поиска единичных
ключей\label{tab:02}}\\
    \hline
    \multicolumn{1}{|c|}{\multirow{2}{*}{\textbf{Метод}}} &
    \multicolumn{2}{c|}{\textbf{Сложность}} &
    \multicolumn{1}{c|}{\multirow{2}{*}{\textbf{Процент коллизий}}}\\
    \cline{2-3}
    & \multicolumn{1}{c|}{\textbf{Худший}}
    & \multicolumn{1}{c|}{\textbf{Средний}}
    &\\
    \hline
    Хеш-индексы
    & $O(n)$
    & $O(1)$
    & $35.3\%$\\
    \hline
    Обученые хеш-индексы
    & $O(n)$
    & $O(1)$
    & $19.5\%$\\
    \hline
\end{longtable}
}

Таким образом, обученные хеш-индексы могут обеспечить уменьшение количества
коллизий на $44.8\%$.


%\imgw{complexity}{h!}{17cm}{Классификация методов по сложности чтения-записи}
%
%{
%\captionsetup{format=hang,justification=raggedright,
%              singlelinecheck=off,width=16.8cm}
%\begin{longtable}[Hc]{|p{3cm}|p{2cm}|p{2cm}|p{2cm}|p{2cm}|p{3.5cm}|}
%\caption{Сравнение методов построения индексов\label{tab:02}}\\
%    \hline
%    \multicolumn{1}{|c}{\multirow{2}{*}{\textbf{Метод}}} &
%    \multicolumn{2}{|c|}{\textbf{Время}} &
%    \multicolumn{2}{c|}{\textbf{Память}} &
%    \multicolumn{1}{c|}{\multirow{2}{*}{\textbf{Применение}}}\\
%    \cline{2-5}
%    & \multicolumn{1}{c|}{\textbf{Чтение}}
%    & \multicolumn{1}{c|}{\textbf{Запись}}
%    & \multicolumn{1}{c|}{\textbf{Чтение}}
%    & \multicolumn{1}{c|}{\textbf{Запись}}
%    &\\
%    \hline
%    B-деревья & $O(\log N)$
%    & $O(\log N)$
%    & $O(\log N)$
%    & $O(\log N)$
%    & \multirow{2}{*}{\parbox{3.5cm}{Поиск ключей в диапазоне}}\\
%    \cline{1-5}
%    Обученные индексы & $O(\log N)$
%    & $O(\log N)$
%    & $O(\log N)$
%    & $O(\log N)$
%    & \\
%    \hline
%    Хеш-индексы & $O(\log N)$
%    & $O(\log N)$
%    & $O(\log N)$
%    & $O(\log N)$
%    & \multirow{2}{*}[-.7ex]{\parbox{3.5cm}{Поиск\newlineединичного\newlineключа}}\\
%    \cline{1-5}
%    Обученные хеш-индексы & $O(\log N)$
%    & $O(\log N)$
%    & $O(\log N)$
%    & $O(\log N)$
%    & \\
%    \hline
%    Фильтр Блума & $O(\log N)$
%    & $O(\log N)$
%    & $O(\log N)$
%    & $O(\log N)$
%    & \multirow{2}{*}[-.7ex]{\parbox{\hsize}{Проверка\newlineсуществования\newlineключа}}\\
%    \cline{1-5}
%    Обученные индексы & $O(\log N)$
%    & $O(\log N)$
%    & $O(\log N)$
%    & $O(\log N)$
%    & \\
%    \hline
%\end{longtable}
%}
