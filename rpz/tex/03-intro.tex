\chapter*{ВВЕДЕНИЕ}
\addcontentsline{toc}{chapter}{ВВЕДЕНИЕ}

На протяжении последнего десятилетия происходит автоматизация все большего числа
сфер человеческой деятельности~\cite{koptenok}. Это приводит к тому, что с
каждым годом производится все больше данных. Так, по исследованию компании
IDC~(International Data Corporation), занимающейся изучением мирового рынка
информационных технологий и тенденций развития технологий, объем данных к
2025~году составит около 175~зеттабайт, в то время как на год исследования их
объем составлял 33~зеттабайта~\cite{idc}.

Для хранения накопленных данных используются базы данных~(БД), доступ к ним
обеспечивается системами управления базами данных~(СУБД), обрабатывающими
запросы на поиск, вставку, удаление или обновление. При больших объемах
информации необходимы методы для уменьшения времени обработки запросов, одним из
которых является построение индексов~\cite{bits}.

Базовые методы построения индексов используют такие структуры, как деревья
поиска, хеш-таблицы и битовые карты~\cite{dama}. На основе данных методов
проводятся исследования по разработке новых для уменьшения времени поиска и
затрат на перестроение индекса при изменении данных, а также сокращения
дополнительно используемой памяти. Одно из таких исследований~\cite{main} было
проведено в 2018 году, авторы которого, опираясь на идею, что обычные индексы не
учитывают распределение данных, предложили новый вид индексов, основанный на
машинном обучении, и назвали их обученные индексы~(learned~indexes). За
последние пять лет было проведено множество исследований~\cite{alex, apex,
ulipp, pgmi} по совершенствованию обученных индексов в плане поддержки операций
и улучшению производительности, поэтому в данной работе в сравнение к базовым
приводятся методы построения обученных индексов.


Целью данной работы является классификация методов построения индексов в
базах данных.

Для достижения поставленной цели требуется решить следующие задачи:
\begin{itemize}
    \item провести анализ предметной области: дать основные определение, описать
        свойства индексов и их типы;
    \item описать методы построения индексов в базах данных;
    \item предложить и обосновать критерии оценки качества описанных методов
      и сравнить методы по предложенным критериям оценки.
\end{itemize}
