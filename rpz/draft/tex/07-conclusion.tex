\chapter*{ЗАКЛЮЧЕНИЕ}
\addcontentsline{toc}{chapter}{ЗАКЛЮЧЕНИЕ}

Подвести к обученным индексам (learned indices) даже раньше заключения.

Бла-бла-бла --- для поиска наилучших характеристик индексов можно использовать
методы машинного обучения.

Мои предположения:

\begin{itemize}
    \item достаточно иметь не полностью сбалансированное дерево поиска и при
        этом не проигрывать во времени доступа, но уменьшать потери при вставке
        и удалении);
    \item выбор с помощью ML структуры индекса (разреженный/неразреженный,
        кластеризованный/некластеризованный);
    \item простая замена всего индекса на предсказывающую положение записи
        обученную модель???
\end{itemize}

Вопросы:

\begin{itemize}
    \item нужно ли описывать вставку и удаление в индексы?
    \item какие структуры описывать (деревья)?
    \item что с ТЗ на диплом?
\end{itemize}

