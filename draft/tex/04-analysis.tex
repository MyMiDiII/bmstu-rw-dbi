\chapter{Анализ предметной области}

Данных много => актуально => базы данных.

База данных --- это ...

Основная операция --- поиск => создание методов для ускорения данной операции,
одним из которых является индексы (\bfit{есть ли другие???}).

Индекс -- это ...

Существует два основных вида индексов \bfit{уточнить???}:

\begin{itemize}
    \item упорядоченные, реализующиеся на основе деревьев поиска;
    \item хеш-индексы, в которых поиск значений осуществляется с помощью
        вычисления хеш-функции.
    \item \bfit{bitmap-индексы??? (индексы на основе битовых карт).}
\end{itemize}

Индекс представляет собой структуру, которая строится в дополнение к
существующим данным. Таким образом, она занимает дополнительный объем памяти и
должна соответствовать текущим данным, то есть необходимо изменять данную
структуру при вставке или удалении элементов. Так как индексы создаются для
осуществления поиска, то они также характеризуются типом и временем доступа.

Таким образом, можно выделить следующие характеристики индексов (\bfit{мб по ним
и оценивать, почему нет}):


\begin{itemize}
    \item \bfit{тип доступа} --- поиск записей по аттрибуту с конкретным
        значением, или со значением из указанного диапазона;
    \item \bfit{время доступа} --- время поиска записи или записей;
    \item \bfit{время вставки}, включающее время поиска правильного места вставки, а
        также время для обновления индекса;
    \item \bfit{время удаления}, аналогично вставке, включающее время на поиск
        удаляемого элемента и время для обновления индекса;
    \item \bfit{дополнительная память}, занимаемая индексной стркутурой.
\end{itemize}

\bfit{Ключ поиска} --- аттрибут или набор аттрибутов, по которым осуществляется
поиск записей.
