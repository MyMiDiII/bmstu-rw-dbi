\documentclass{bmstu-pr}

\begin{document}

\prtitle{Классификация методов построения\\индексов в базах данных}{Маслова
Марина Дмитриевна}{ИУ7-73Б}{Оленев Антон Александрович}

\begin{frame}
    \frametitle{Цель и задачи}

    \textbf{Цель:} классификация методов построения индексов в
    базах данных.

    \textbf{Задачи:}
    \begin{itemize}
        \item провести анализ предметной области: дать основные определение, описать
            свойства индексов и их типы;
        \item описать методы построения индексов в базах данных;
        \item предложить и обосновать критерии оценки качества описанных методов
          и сравнить методы по предложенным критериям оценки.
    \end{itemize}
\end{frame}

\begin{frame}
    \frametitle{Основные определения}

    Индекс --- это некоторая структура, обеспечивающая быстрый поиск записей в
    базе данных.

    Индекс:

    \begin{itemize}
        \item определяет соответствие ключа поиска конкретной записи с
            положением этой записи;
        \item строится в дополнение к существующим данным;
        \item описывается:
            \begin{itemize}
                \item типом и временем доступа;
                \item временем вставки и удаления;
                \item дополнительной памятью, занимаемая индексной структурой.
            \end{itemize}
    \end{itemize}
\end{frame}

\begin{frame}
    \frametitle{Типы индексов}

    \begin{itemize}
        \item кластеризованные и некластеризованные;
        \item плотные и разреженные;
    \end{itemize}

    \hspace*{\fill}%
    \raisebox{-\height}{\imgs{dense}{1.2}}%
    \hfill
    \raisebox{-\height}{\imgs{sparse}{1.2}}%
    \hspace*{\fill}

\end{frame}

\begin{frame}
    \frametitle{Типы индексов}

    \begin{itemize}
        \item одноуровневые и многоуровневые.
    \end{itemize}
    ~\\

    \centering\imgs{multilevel}{1.2}
\end{frame}

\begin{frame}
    \frametitle{Типы индексов}

    \begin{itemize}
        \item индексы на основе деревьев поиска;
        \item индексы на основе хеш-таблиц;
        \item индексы на основе битовых карт.
    \end{itemize}

\end{frame}

\begin{frame}
    \frametitle{B-деревья}

    \centering\imgs{node}{1.2}
\end{frame}

\begin{frame}
    \frametitle{Построение B-деревьев}
    \centering\imgs{rootInsert}{1.2}
\end{frame}

\begin{frame}
    \frametitle{Построение B-деревьев}
    \centering\imgs{leafInsert}{1}
\end{frame}

\begin{frame}
    \frametitle{B$^+$-деревья}
    \centering\imgs{bplustree}{1.2}
\end{frame}

\begin{frame}
    \frametitle{Обученные индексы}
    \centering\imgs{blearnedcomp}{1.5}
\end{frame}

\begin{frame}
    \frametitle{Рекурсивная модель}
    \centering\imgs{rmi}{1.5}
\end{frame}

\begin{frame}
    \frametitle{Хеш-индексы}
    \centering\imgs{buckethash}{1.5}
\end{frame}

\begin{frame}
    \frametitle{Обученные хеш-индексы}
    \centering\imgs{hashlearnedhash}{1.5}
\end{frame}

\begin{frame}
    \frametitle{Фильтр Блума и обученные индексы}
\end{frame}

\begin{frame}
    \frametitle{Классификация}
\end{frame}

\begin{frame}
    \frametitle{Классификация}
\end{frame}

\begin{frame}
    \frametitle{Классификация}
\end{frame}

\begin{frame}
    \frametitle{Классификация}
\end{frame}

\begin{frame}
    \frametitle{Заключение}
\end{frame}

\end{document}
