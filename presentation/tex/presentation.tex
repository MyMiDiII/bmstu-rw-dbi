\documentclass{bmstu-pr}

\begin{document}

\prtitle{Классификация методов построения\\индексов в базах данных}{Маслова
Марина Дмитриевна}{ИУ7-73Б}{Оленев Антон Александрович}

\begin{frame}
    \frametitle{Цель и задачи}

    \textbf{Цель:} классификация методов построения индексов в
    базах данных.

    \textbf{Задачи:}
    \begin{itemize}
        \item провести анализ предметной области: дать основные определение, описать
            свойства индексов и их типы;
        \item описать методы построения индексов в базах данных;
        \item предложить и обосновать критерии оценки качества описанных методов
          и сравнить методы по предложенным критериям оценки.
    \end{itemize}
\end{frame}

\begin{frame}
    \frametitle{Основные определения}

    Индекс --- это некоторая структура, обеспечивающая быстрый поиск записей в
    базе данных.

    Индекс:

    \begin{itemize}
        \item определяет соответствие ключа поиска конкретной записи с
            положением этой записи;
        \item строится в дополнение к существующим данным;
        \item описывается:
            \begin{itemize}
                \item типом и временем доступа;
                \item временем вставки и удаления;
                \item дополнительной памятью, занимаемая индексной структурой.
            \end{itemize}
    \end{itemize}
\end{frame}

\begin{frame}
    \frametitle{Типы индексов}

    \begin{itemize}
        \item кластеризованные и некластеризованные;
        \item плотные и разреженные;
    \end{itemize}

    \hspace*{\fill}%
    \raisebox{-\height}{\imgs{dense}{1.2}}%
    \hfill
    \raisebox{-\height}{\imgs{sparse}{1.2}}%
    \hspace*{\fill}

\end{frame}

\begin{frame}
    \frametitle{Типы индексов}

    \begin{itemize}
        \item одноуровневые и многоуровневые.
    \end{itemize}
    ~\\

    \centering\imgs{multilevel}{1.2}
\end{frame}

\begin{frame}
    \frametitle{Типы индексов}

    \begin{itemize}
        \item индексы на основе деревьев поиска;
        \item индексы на основе хеш-таблиц;
        \item индексы на основе битовых карт.
    \end{itemize}

\end{frame}

\begin{frame}
    \frametitle{B-деревья}

    \centering\imgs{node}{1.2}
\end{frame}

\begin{frame}
    \frametitle{Построение B-деревьев}
    \centering\imgs{rootInsert}{1.2}
\end{frame}

\begin{frame}
    \frametitle{Построение B-деревьев}
    \centering\imgs{leafInsert}{1}
\end{frame}

\begin{frame}
    \frametitle{B$^+$-деревья}
    \centering\imgs{bplustree}{1.2}
\end{frame}

\begin{frame}
    \frametitle{Обученные индексы}
    \centering\imgs{blearnedcomp}{1.5}
\end{frame}

\begin{frame}
    \frametitle{Рекурсивная модель}
    \centering\imgs{rmi}{1.5}
\end{frame}

\begin{frame}
    \frametitle{Хеш-индексы}
    \centering\imgs{buckethash}{1.5}
\end{frame}

\begin{frame}
    \frametitle{Обученные хеш-индексы}
    \centering\imgs{hashlearnedhash}{1.5}
\end{frame}

\begin{frame}
    \frametitle{Фильтр Блума и обученные индексы}
    \imgfs{bloom}{h!}{1.5}
    \imgfs{learnedBloom}{h!}{1.5}
\end{frame}

\begin{frame}
    \frametitle{Классификация}

Классификация методов построения индексов для поиска ключей, принадлежащих
некоторому диапазону (поиск).
{
\captionsetup{format=hang,justification=raggedright,
              singlelinecheck=off,width=17cm}
\begin{longtable}[Hc]{|p{8cm}|p{2cm}|p{2cm}|p{4.8cm}|}
    \hline
    \multicolumn{1}{|c|}{\textbf{Метод}} &
    \multicolumn{1}{c|}{\textbf{Сложность}} &
    \multicolumn{1}{c|}{\textbf{Время, нс}} &
    \multicolumn{1}{c|}{\parbox{4.8cm}{\vspace{2mm}\centering\textbf{Обращения к
    памяти}}}\\[2.5ex]
    \hline
    B-деревья & $O(\log N)$
    & $237.94$
    & $57.0$\\
    \hline
    Обученные индексы & $O(\log N)$
    & $139.09$
    & $12.6$\\
    \hline
    LIPP & $O(\log N)$
         & \color{white}$1$\color{black}$24.23$
    & \color{white}$1$\color{black}$3.1$\\
    \hline
\end{longtable}
}

\end{frame}

\begin{frame}
    \frametitle{Классификация}
Классификация методов построения индексов для поиска ключей,
принадлежащих некоторому диапазону (вставка)
{
\captionsetup{format=hang,justification=raggedright,
              singlelinecheck=off,width=16.8cm}
\begin{longtable}[Hc]{|p{8cm}|p{2cm}|p{2cm}|p{4.8cm}|}
    \hline
    \multicolumn{1}{|c|}{\textbf{Метод}} &
    \multicolumn{1}{c|}{\textbf{Сложность}} &
    \multicolumn{1}{c|}{\textbf{Время, нс}} &
    \multicolumn{1}{c|}{\parbox{4.8cm}{\vspace{2mm}\centering\textbf{Обращения к
    памяти}}}\\[2.5ex]
    \hline
    B-деревья
    & $O(\log N)$
    & $1114.19$
    & $57.8$\\
    \hline
    Обученные индексы
    & ---
    & ---
    & ---\\
    \hline
    LIPP
    & $O(\log^2 N)$
    & \color{white}$11$\color{black}$70.93$
    & \color{white}$1$\color{black}$3.1$\\
    \hline
\end{longtable}
}
\end{frame}

\begin{frame}
    \frametitle{Классификация}
Классификация методов построения индексов для поиска единичных
ключей
{
\captionsetup{format=hang,justification=raggedleft,
              singlelinecheck=off,width=13.3cm}
\begin{longtable}[Hc]{|p{6.5cm}|p{2cm}|p{2cm}|p{4.8cm}|}
    \hline
    \multicolumn{1}{|c|}{\multirow{2}{*}{\textbf{Метод}}} &
    \multicolumn{2}{c|}{\textbf{Сложность}} &
    \multicolumn{1}{c|}{\multirow{2}{*}{\parbox{4.8cm}{\textbf{Процент\newlineколлизий}}}}\\
    \cline{2-3}
    & \multicolumn{1}{c|}{\textbf{Худший}}
    & \multicolumn{1}{c|}{\textbf{Средний}}
    &\\
    \hline
    Хеш-индексы
    & $O(N)$
    & $O(1)$
    & $35.3\%$\\
    \hline
    Обученные хеш-индексы
    & $O(N)$
    & $O(1)$
    & $19.5\%$\\
    \hline
\end{longtable}
}
\end{frame}

\begin{frame}
    \frametitle{Классификация}
Классификация методов построения индексов для проверки существования
ключа в наборе данных
{
\captionsetup{format=hang,justification=raggedleft,
              singlelinecheck=off,width=12cm}
\begin{longtable}[Hc]{|p{8cm}|p{2cm}|p{2cm}|}
    \hline
    \multicolumn{1}{|c|}{\textbf{Метод}} &
    \multicolumn{1}{c|}{\textbf{Сложность}} &
    \multicolumn{1}{c|}{\textbf{Размер, MБ}}\\
    \hline
    Фильтр Блума
    & $O(k)$
    & $2.04$\\
    \hline
    Обученные индексы
    & $O(1)$
    & $1.31$\\
    \hline
\end{longtable}
}
\end{frame}

\begin{frame}
    \frametitle{Заключение}
    В ходе данной работы:
    \begin{itemize}
        \item проведен анализ предметной области;
        \item описаны методы построения индексов в базах данных;
        \item предложены и обоснованы критерии оценки качества описанных методов
          и проведено сравние методов по предложенным критериям оценки.
    \end{itemize}
    ~\\

    Поставленная цель достигнута.
\end{frame}

\end{document}
